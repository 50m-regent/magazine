\documentclass{jsarticle}

\title{Flutterであなたもアプリデベロッパー!!}
\author{}
\date{}

\begin{document}
    \maketitle
    \tableofcontents

    \section*{はじめに}
        この文章は、筆者が最近アプリ開発を久しぶりにかじり出したら楽しくなっちゃった結果生み出された駄文です。
        心してお読みください。

        この文章を通じて、Flutterというフレームワーク\footnote{システム開発を簡単にできるように用意されたプログラム}
        を使用して基礎的なアプリ開発ができるようになることを目標としています。想定している読者は以下の通りです。

        \begin{itemize}
            \item アプリ開発をしたことないけどしてみたい人。
            \item XcodeやAndroid Studioを使ったアプリ開発はめんどくさい人。
            \item 暇な人。
        \end{itemize}

        逆に、今までアプリ開発をいろいろやってきて、「新しいフレームワークでも調べながら
        ごりごりいじれるよ!」みたいな人は想定していません。お帰りください。

        この文章では、基本的な専門用語も脚注などで軽い説明を入れてあるつもりです。
        わからないところが出てきても、とりあえず少し先まで読んでみてから調べることをお勧めします。

        また、本書で使用しているコードは筆者のGitHub(https://github.com/50m-regent/magazine)で
        全て公開しているので、ぜひ参考にしてください。

        最後に、本書に関してご意見などありましたら筆者のTwitter(@50m\_regent)までご意見を
        いただければ幸いです。

        では、Flutterの世界に飛び込みましょう!

    \section*{Flutterって何?}
        皆さんはGoogleという会社を知っているでしょうか。
        むしろ知らない人がいるのかというレベルですね。
        言わずと知れた世界の大企業です。
        そんなGoogle
    
\end{document}